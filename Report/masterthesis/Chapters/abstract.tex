{\chapter*{Abstract}}

The rapidly progressing digital revolution have infused enormous amount of images and video, which is growing constantly. Dealing with this large scale data calls for software application for enhancing image quality, classification, segmentation and recognition is the topic of Image Processing. In any image processing task an image is usually represented as a vector by concatenating each row or column. The dimension of the image vector is very high and typically embedded on a non-linear dimensional manifold, whose dimension is much smaller than that of the original data space. Manifold learning algorithms extract a submanifold embedded in a high-dimensional space, thus widely used in in the field of computer vision for dimensionality reduction, noise handling, classification and segmentation. 

In the first part of this thesis, we provide mathematical notions necessary for under standing the intuition behind manifold learning. After introducing the necessary tools, we review the representative sample of manifold learning, mathematical developments, as well as some interesting applications.

The second part of thesis focuses on application of manifold learning in optical character recognition, in particular MNIST database of handwritten digits recognition. We use representative manifold learning algorithms together with supervised learning model such as KNN, SVM and CNN to classify digit with high accuracy.

In the spirit of manifold learning algorithms, the third part of this thesis tackles the problem of anomaly detection in image. We use dimension reduction property of diffusion maps together with multiscale approach based on Laplacian pyramid representation to detect anomaly from background pixels.


